\documentclass[a4paper, titlepage]{article}
\usepackage[round, sort, numbers]{natbib}
\usepackage[utf8]{inputenc}
\usepackage{amsfonts, amsmath, amssymb, amsthm}
\usepackage{color}
\usepackage{listings}
\usepackage{mathtools}
\usepackage{multicol}
\usepackage{paralist}
\usepackage{parskip}
\usepackage{subfig}
\usepackage{tikz}
\usepackage{titlesec}

\numberwithin{figure}{section}
\numberwithin{table}{section}

\usetikzlibrary{arrows, automata, backgrounds, petri, positioning}
\tikzstyle{place}=[circle, draw=blue!50, fill=blue!20, thick]
\tikzstyle{marking}=[circle, draw=blue!50, thick, align=center]
\tikzstyle{transition}=[rectangle, draw=black!50, fill=black!20, thick]

% define new commands for sets and tuple
\newcommand{\setof}[1]{\ensuremath{\left \{ #1 \right \}}}
\newcommand{\tuple}[1]{\ensuremath{\left \langle #1 \right \rangle }}
\newcommand{\card}[1]{\ensuremath{\left \vert #1 \right \vert }}

\makeatletter
\newcommand\objective[1]{\def\@objective{#1}}
\newcommand{\makecustomtitle}{%
	\begin{center}
		\huge\@title \\
		[1ex]\small Dimitri Racordon \\ \@date
	\end{center}
	\@objective
}
\makeatother

\begin{document}

  \title{Outils formels de Modélisation \\ 10\textsuperscript{ème} séance d'exercices}
  \author{Dimitri Racordon}
  \date{01.12.17}
	\objective{
    Dans cette séance d'exercices,
    nous allons manipuler des preuves en utilisant la théorie des séquents.
  }

	\makecustomtitle

  \section{Preuves par séquents ($\bigstar\bigstar$)}
    En utilisant la méthode des séquents, prouvez les formules suivantes:
    \begin{enumerate}
      \item $A \land B \vdash A \vee B$
      \item $A \vee B \vdash (A \land C) \vee (B \land C) \vee \neg C$
      \item $A \Rightarrow B, A \Rightarrow C, B \Rightarrow D, C \Rightarrow D \vdash A \Rightarrow D$
      \item $A \Rightarrow B, B \Rightarrow A \vdash A \Leftrightarrow B$
    \end{enumerate}

  \section{Fainéantise ($\bigstar\bigstar\bigstar$)}
    Nous disposons des propositions suivantes:
    \begin{compactitem}
      \item S'il fait beau, on va à l'université.
      \item S'il risque de pleuvoir, on ne va pas à l'université.
      \item Si on n'est pas à la maison, on est à l'université.
      \item Si on reste à la maison, on fait des exercices et on relit le cours.
      \item Si le ciel est gris, il risque de pleuvoir.
      \item Si on va à l'université, on fait des exercices et on est contents.
      \item Soit il fait beau, soit le ciel est gris.
    \end{compactitem}

    Démontrez en utilisant des séquents que, dans un monde décrit par de telles affirmations,
    on est condamné à toujours faire des exercices.
    Commencez par les exprimer sous forme de formules propositionnelles,
    créez un jugement approprié, puis démontrez-le.

\end{document}
